\documentclass[12pt]{report}%
\usepackage[utf8]{inputenc}%
\usepackage[T1]{fontenc}%
\usepackage[UKenglish]{babel,datetime}%english gestion
\usepackage{hyperref}
\usepackage[bottom]{footmisc}
\usepackage[font=small,labelfont=bf]{caption}
\usepackage[newparttoc]{titlesec}
\usepackage[export]{adjustbox}
\usepackage[margin=0.95in]{geometry}

%All the packages
\usepackage{lmodern,ragged2e,textcomp,lmodern}
\usepackage{graphicx,xcolor,float,fancyhdr}
\usepackage{chngcntr,times,lastpage,setspace}
\usepackage{subcaption,wrapfig,fancyhdr,blindtext}
\usepackage{titletoc,afterpage,transparent}


%Nouvelles commandes
\newcommand{\periodafter}[1]{#1 -}
\newcommand{\HRule}{\rule{\linewidth}{0.5mm}}
\newcommand{\Mline}{\hrule \mbox{}\\[0.1cm]}
\renewcommand{\thechapter}{\Roman{chapter}}%Changing chapter numbering
\renewcommand*\thesection{\arabic{section}}%Changin section numbering
\newcommand\blankpage{%
    \null
    \thispagestyle{empty}%
    \addtocounter{page}{-1}%
    \newpage}

%FORMAT DU CHAPITRE (table of content compris)
\titleclass{\chapter}{top}
\titleformat{\chapter}[hang]
  {}
  {\normalfont \bfseries \large \thechapter. \hspace{0.2cm} }
  {0pt}
  {\normalfont \large \bfseries }
  [{\color{gray}\titlerule[1pt]}]
\titlespacing*{\chapter}{0pt}{50pt}{18pt}


%Format section
\titleclass{\section}{straight}
\titleformat{\section}[hang]
  {}
  {\normalfont \hspace{1cm} \bfseries \thesection. \hspace{0.5cm} }
  {0pt}
  {\normalfont \bfseries }
\titlespacing*{\section}{0pt}{50pt}{18pt}

%Format subsection
\titleclass{\subsection}{straight}
\titleformat{\subsection}[hang]
  {}
  {\normalfont \hspace{2cm} \bfseries \thesubsection. }
  {0pt}
  {\normalfont \bfseries }
\titlespacing*{\subsection}{0pt}{40pt}{10pt}

%NEW DATE FORMAT
\newdateformat{ComputerDate}{\THEDAY /\THEMONTH /\THEYEAR}

%CHANGE LINE SPACING
\renewcommand{\baselinestretch}{1.5} 

%CHANGE FIGURE NUMBERING
\renewcommand\thefigure{\arabic{figure}}

%PREVENT FIGURES FROM RESETING EACH CHAPTER
\counterwithout{figure}{chapter}


\begin{document}

\begin{titlepage}
\begin{center}
 \fbox{\includegraphics[width=0.4\textwidth]{images/ITB.png}}\\[0.4cm]

{ \Huge \textbf{ASSIGNMENT COVER SHEET} \\[0.4cm] }
\end{center}\mbox{}\\[1cm]
\begin{Large}
{\setstretch{1.5}
\textbf{Students Name and ID : }Pierre Thubé (B00092354) -  Azarias Boutin (B00092351) \\
\textbf{Course : } Rich Web application\\
\textbf{Year : } Third year\\
\textbf{Lecturer : } Orla McMahon \\
\textbf{Title of assignment : } Rich Web Applications : Assignment 2 \\
\textbf{Due Date : } 6th December 2015 \\
\textbf{Date Submitted :} \today  \\
\par}

\end{Large}

\begin{justify}
The material contained in this assignment is the authors original work, except where work quoted is duly acknowledged in the text. No aspect of this assignment has been previously submitted for assessment in any other unit or course.  
\end{justify}
\end{titlepage}

\setcounter{page}{2}

\chapter*{Abstract}
This report explain the steps in the development of a web application for desktop which is adaptable for other device like smartphones or tablets, from the conceptual planning and design to the realisation. This application has been realized in group of two, dividing the work according to forces and weaknesses of each. This website had to be a professional and engaging web application that will enhance the user experience when using it. 

\tableofcontents

\listoffigures


\chapter{Application}
\section{Technologies}
We used HTML, CSS, JavaScript, jQuery mobile and JSON to realize the application.\\
For the three main sections, we used one API for each. For the blog, we used Wordpress's API to set up a blog. To be able to get the blog's posts, we installed a JSON plug-in.\\
For the section called "Pictures", we used Flickr's API to display photos from a Flickr account.\\
For the Google Map section, we used the Google Map's API.
\section{Design}
\subsection{Form}
The application is composed of a header, a content part and a footer. On the content page you can find four different sections, each one display something different.\\
There is a responsive CSS. For each kind of support (desktop, tablet, smart-phones,...) the CSS will be adapted depending on the size of the screen.
\subsection{Background}
For the design, we chose a background pictures related to our subject, houses of champagne in France. We used brown colors (wood imitation) and when mouse over a section, this last one keep its original color but the other ones change and turn grey.
\begin{figure}[h]
    \begin{minipage}[c]{.45\linewidth}
        \centering
        \includegraphics[scale=0.18]{images/normal.png}
        \caption{\it Normal layout}
    \end{minipage}
    \hfill%
    \begin{minipage}[c]{.45\linewidth}
        \centering
        \includegraphics[scale=0.18]{images/mouse_hover.png}
        \caption{\it Mouse hover on section pictures}
    \end{minipage}
\end{figure} 
\subsection{jQuery Theme}
For the jQuery theme, we used the jQuery theme roller to customize the colors. We chose colors in relation with nature because we thought this was pretty beautiful and related to Champagne. Thus, you can find on the application,  green background and green buttons.
\subsection{Pictures}
Pictures are display in grid format. We used CSS code which create the grid in function of the size of each picture because all pictures doesn't have the same original size, thereby the CSS code create the grid taking account of the size and doesn't alter the quality of each pictures keeping the scale of pictures.

\chapter{Who did what}
We tried to separate work as equal as possible but also according to our skills and ability. Azarias realized the WordPress part and Pierre the Flickr part. For the rest of the project, we tried to take decisions together. As Azarias has more abilities, he probably produced more code than Pierre indeed this last one spent more time than Azarias on the report. At the end, we both produced code and wrote the report. 

\chapter{What we learned from the project}
We both discovered WordPress and Flickr for the first time. We have never used them before so we learned how it was working and how to use it.\\
Azarias has more skills than Pierre in javascript, so for Azarias it was not new but for Pierre, following the first Assignment and the discover of javascript, this project allowed him to learn more javascript.\\
This also allowed us to learn how to adapt CSS according to the size of the screen used by users. 

\chapter{Research}
\section{Existing JavaScript frameworks and libraries}
In today's world, javascript is one of the most used language. To help developers to be more efficient and to produce maintainable code a lots of frameworks and libraries are being developed. 

\subsection{Most common ones}
Here is a list of the most famous JavaScript frameworks and libraries,a short description,  the strength and weakness of each.

\begin{description}
	\item[Angular.js] : Developped by Google, the DOM is directly linked with the data. Follow the MVW (model-view-whatever) architectural pattern. Is open source and available on Github. \begin{description}
		\item[Strengths] : automatic UI refreshing. MVW pattern make it easier to maintain.
		\item[Weaknesses] : still in development is really complex when going a bit deeper.
	\end{description}
	\item[Backbone.js] : Developped by a community. Help to interact with the DOM and give access to function to facilitate data processing. Is also open source and available on Github. \begin{description}
		\item[Strengths] : Really useful when a lot of data processing is required. (Thanks to underscore.js)
		\item[Weaknesses] : DOM manipulation is still at a low level.
	\end{description}
	\item[Knockout.js] : To provide fully interactive website. Permit high level DOM manipulation by binding the DOM to the model. \begin{description}
	\item[Strengths] : Hight level DOM manipulation. Automatic UI refreshing
	\item[Weaknesses] : Only useful for DOM manipulation. Does not help go structure the code. 
	\end{description}
	\item[Knockback.js] : Fusion of Backbone.js and knockout.js \begin{description}
		\item[Strengths] : strengths of both backbone and knockout.js : ease data processing,routing and hight level DOM manipulation with automatic UI refreshing
		\item[Weaknesses] : Really heavy.
	\end{description}
	\item[Vue.js] : DOM manipulation at the highest level. Concentrate on DOM manipulation from data. Usefull to create super interactive websites. \begin{description}
		\item[Strengths] : Make it really easier to manipulate DOM
		\item[Weakness] : Used only to manipulate DOM
	\end{description}
\end{description}

There are a lost of others javascript framework, but some of them also have a single goal (e.g. Raphaël.js for SVG manipulation).


\subsection{Mobile application related ones}
The previous frameworks and libraries are not optimised for mobile applications development.\\
Here is a list of some existing alternatives to jquery Mobile.

\begin{description}
	\item[Cordova] : More than just create an mobile application, it enable the develper to compile and install it on the most commons mobile platforms. i.e. Android,IOS, Windows phone, Blackberry.
	\item[Ratchet] : Created by twitter Bootstrap's developers. It comes with a collection of User Interface and JavaScript plugins for building simple mobile applications, providing reusable HTML classes.
	\item[Ionic] : Framework to create mobile application with high performances. It works best together with Angular.js to build an interactive app. Similar to Ratchet, Ionic is shipped with a nicely crafted font icon set, Ionicons, and a bunch of reusable HTML classes to build the mobile UI.
	\item[Lungo] : a lightweight mobile framework based on HTML5 and CSS3. Lungo brings a number of JavaScript API to control the app.
\end{description}

\section{Selected framework : Ratchet}
Instead of using jQuery mobile for this project, we could have used ratchet. Which is the framework developed by twitter bootstrap's team.\\
Ratchet looks a lot like jQuery mobile. All the application stands on one single page. This page is divided into differents \textit{cards}.
Here is a list of the features/components that ratchet offers :
\begin{description}
	\item[Bar] : a bar at the top or at the bottom of the page. It is possible to add multiple differents features on the bar. Such as  labels, tabs,buttons, icons and more.
	\item[Tables] : the table are stylized and can be used as list. The tables are fully customizable. The developers can add icons, buttons, images, captions, badges.
	\item[Button] : there are four differents types of buttons. The default one, the primary one, the success one and the warning one. These differents type are inspired from bootstrap itself. As every components, it is possible to customize the buttons and add captions, icons and badges on it. The button can also be grouped to hold on a single line.
	\item[Forms] : also inspired from bootstrap, some pre-existing class where made to create more user-friendly forms.
	\item[Popover] : still inspired directly from bootstrap : a popover can be easily created with ratchet. The popover will darken the page and pop over the page.
	\item[Modal] : the modal is like a popover with the difference that it will cover the entire page (instead of only pop and darken the background page).
	\item[Slider] : to go from a page/picture to another by sliding the current page. Instead of pressing a button, using the finger movement to change of content.
	\item[Push] : another way to load a page. Instead of having directly all the html directly on the main file, it injects it when the user push on a button.
\end{description}
To put it in a nutshell : this framework is well-thought, and inspired from the past of bootstrap. There are a lost of common features with jQuery mobile but also some more features like the slider and the push features.

\chapter{Conclusion}
To conclude, we had to realize within our studies a web application for all size of support (desktop, tablets or smart-phones). This application allowed us to discover new APIs like WordPress's and Flickr's, to improve our skills in CSS,HTML and Javascript or at least to revise those languages. We also discover thanks to the research part new frameworks and libraries.
\clearpage
%Breakdown


\end{document}
